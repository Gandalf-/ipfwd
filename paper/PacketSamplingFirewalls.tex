\documentclass[fleqn]{article}%
\usepackage{amsmath}%
\usepackage{amsfonts}%
\usepackage{amssymb}%
\usepackage{graphicx}
%-------------------------------------------
\newtheorem{theorem}{Theorem}
\newtheorem{acknowledgement}[theorem]{Acknowledgement}
\newtheorem{algorithm}[theorem]{Algorithm}
\newtheorem{axiom}[theorem]{Axiom}
\newtheorem{case}[theorem]{Case}
\newtheorem{claim}[theorem]{Claim}
\newtheorem{conclusion}[theorem]{Conclusion}
\newtheorem{condition}[theorem]{Condition}
\newtheorem{conjecture}[theorem]{Conjecture}
\newtheorem{corollary}[theorem]{Corollary}
\newtheorem{criterion}[theorem]{Criterion}
\newtheorem{definition}[theorem]{Definition}
\newtheorem{example}[theorem]{Example}
\newtheorem{exercise}[theorem]{Exercise}
\newtheorem{lemma}[theorem]{Lemma}
\newtheorem{notation}[theorem]{Notation}
\newtheorem{problem}[theorem]{Problem}
\newtheorem{proposition}[theorem]{Proposition}
\newtheorem{remark}[theorem]{Remark}
\newtheorem{solution}[theorem]{Solution}
\newtheorem{summary}[theorem]{Summary}
\newenvironment{proof}[1][Proof]{\textbf{#1.} }{\ \rule{0.5em}{0.5em}}
\setlength{\textwidth}{7.0in}
\setlength{\oddsidemargin}{-0.35in}
\setlength{\topmargin}{-0.75in}
\setlength{\textheight}{9.2in}
\setlength{\parindent}{0.0in}
\begin{document}

\begin{center}
  \textbf{Packet Sampling Firewalls \\ Austin Voecks}\\
\end{center}

\textbf{Outline}
\begin{itemize}
    \item Abstract
    \item IPFWD
    \item IPFW
    \item Benchmarking Performance
    \item Results
    \item Security Implications
    \item Related Work
    \item Conclusion
    \item References
\end{itemize}

\textbf{Abstract} \\
When using a firewall system like IPFW to detect threats, we can end up doing a
lot of packet processing. This can negatively impact performance-sensitive
systems such as storage nodes. This presentation describes a practical solution
to this problem using a load-weighted probabilistic mechanism that allows a
trade-off between perfect visibility of incoming packets and reduced impact to
system load.

\textbf{IPFWD} \\ 

IPFWD is a daemon for FreeBSD that updates an early rule in
IPFW that has a chance to accept any packet. The probability of acceptance is
updated over time and dependent on the current system load. To account for
this, IPFWD extends IPFW logs to include the likelihood an undetected policy
violation occurred.

This approach is based on the premise that firewall performance can be
improved by reducing the number of rules applied to each packet. Supposing a
whitelist policy, a firewall has to apply every rule to each packet before it's
denied. With IPFWD, we have a chance to accept any packet early and skip any
further computation. Test results shows this reduces the system resources
required to handle the same amount of traffic. 

This is a shift in mindset from typical firewalls. Instead of enforcing every
part of a security policy all the time, IPFWD enforces the policy some of the
time and extrapolates from the violations it encounters. This provides insight
into the violations that weren't detected. For this cost, you gain increased
firewall performance and network throughput in resource bound systems. It's
acceptable to allow a percentage of policy violations given that network
traffic patterns are often repeated and the goal is detection and not immediate
prevention. Preventative action may be taken later when resource requirements
are lower.

As an example, under heavy load IPFWD may immediately accept 40\% of packets.
Supposing a port scan was initiated during this time and rules exist to block
it, 60\% of the port scan would still be rejected and logged. Since the
probability will fluctuate over time, IPFWD provides information in the IPFW
logs to show the chance undetected violations occurred for each detected
violation. IPFWD allows administrators to keep extensive rule sets that fully
implement their security policy. Instead of having to simplify rule sets to
increase performance, IPFWD balances policy enforcement and performance
automatically. Under normal or light load, IPFWD will enforce the entire
security policy 100\% of the time. 

\end{document}
